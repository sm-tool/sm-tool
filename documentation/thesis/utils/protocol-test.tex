\newpage
\begin{center}
    \huge Protokół Testowania Aplikacji\\
    \large do Modelowania Sytuacji Kryzysowych\\
\end{center}
\section{Cel Testowania}
Przeprowadzenie testów funkcjonalnych aplikacji służącej do modelowania scenariuszy sytuacji kryzysowych stanowi kluczowy element procesu weryfikacji systemu. Oprogramowanie zostało zaprojektowane w celu umożliwienia efektywnego planowania i przeprowadzania ćwiczeń procedur reagowania kryzysowego, a także weryfikacji skuteczności nowo opracowanych procedur operacyjnych. System ma szczególne znaczenie w kontekście przygotowania służb do działań w sytuacjach wymagających skoordynowanej odpowiedzi wielu jednostek. Planowane testy mają na celu potwierdzenie poprawności implementacji wszystkich kluczowych funkcjonalności oraz weryfikację intuicyjności interfejsu użytkownika.
\section{Zakres Testowania}
W ramach procesu testowego przeprowadzona zostanie weryfikacja możliwości modelowania scenariusza gaszenia pożaru w opuszczonym budynku wielopiętrowym. Testowanie obejmuje pełen cykl tworzenia scenariusza, od definicji podstawowych parametrów, przez konfigurację zasobów, aż po szczegółowe modelowanie sekwencji działań. Szczególna uwaga zostanie poświęcona weryfikacji poprawności interakcji między obiektami oraz zachowaniu logicznej spójności scenariusza.
\section{Scenariusz Testowy}
\subsection{Cel Operacyjny}
Głównym założeniem scenariusza jest przeprowadzenie skoordynowanej akcji zabezpieczenia i gaszenia budynku przy optymalnym wykorzystaniu dostępnych zasobów. Scenariusz zakłada współdziałanie dwóch jednostek strażackich wyposażonych w specjalistyczny sprzęt gaśniczy. Kluczowym aspektem jest zachowanie procedur bezpieczeństwa przy jednoczesnym dążeniu do minimalizacji czasu realizacji zadania.

\subsection{Dostępne Zasoby}
W scenariuszu wykorzystany zostanie 8-piętrowy budynek z określoną lokalizacją pożaru, dwie drużyny strażackie wyposażone w specjalistyczny sprzęt oraz pojazd strażacki.

\subsection{Wymagania Scenariusza}
Realizacja scenariusza wymaga przeprowadzenia sekwencji czterech akcji: dwóch związanych z zabezpieczeniem budynku oraz dwóch dotyczących procesu gaszenia, przy czym te ostatnie wymagają aktywnego wykorzystania sprzętu strażackiego.
\section{Metodologia Weryfikacji}
Weryfikacja poprawności modelowania zostanie przeprowadzona manualnie przez wyznaczonego kontrolera, zgodnie z przyjętymi procedurami testowymi. Proces weryfikacji obejmie wszystkie aspekty zdefiniowane w kryteriach akceptacji.
\section{Kryteria Akceptacji}
Podstawą pozytywnej weryfikacji będzie prawidłowe odwzorowanie scenariusza w systemie z zachowaniem wszystkich zależności między obiektami oraz poprawną sekwencją działań operacyjnych. Szczególna uwaga zostanie zwrócona na logiczne powiązania między jednostkami a wykorzystywanym przez nie sprzętem.
\section{Uwagi Końcowe}
Przeprowadzone testy mają kluczowe znaczenie dla potwierdzenia gotowości systemu do praktycznego wykorzystania w procesie planowania działań kryzysowych. Szczególna uwaga zostanie zwrócona na aspekty użytkowe oraz potencjalne obszary wymagające optymalizacji.

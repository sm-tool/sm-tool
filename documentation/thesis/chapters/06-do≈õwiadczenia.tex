\chapter{Doświadczenia}

\section{Praca w zespole}

Praca w zespole przebiegała bez większych problemów. Kontakt zespołu deweloperów z project managerami był naprawdę dobry, 
co skutkowało efektywniejszą pracą oraz dobrym zrozumieniem wymagań zgromadzonych w etapie planowania projektu. Pomimo jasnego
podziału zadań, deweloperzy starali się wymieniać wiedzą, oraz byli otwarci na propozycje zmian od innych członków 
zespołu. Szerokie wykorzystanie platform takich jak Discord lub Google meet pozwoliło nam na bezpośredni kontakt, 
oraz wzajemną pomoc w razie napotkanych problemów. Jakiekolwiek wątpliwości rozwiązywane były poprzez grupowe rozmowy.
Dobre zarządzanie oraz profesjonalizm wszystkich uczestników projektów znacznie ułatwił jego realizację.

\section{Praca z klientem}

Ważną częścią tego projektu była współpraca z klientem, czyli zewnętrzną jednostką która miała oczekiwania co do produktu który 
tworzyliśmy. Najcenniejszym doświadczeniem było przekazanie naszej wizji projektu innym osobom. Podczas rozwoju aplikacji napotkaliśmy
kilka kwestii, głównie dotyczących schematu bazy danych, w których zauważyliśmy opcję zmiany lub poprawy uzgodnionego rozwiązania. 
Po wymianie wielu trafnych argumentów z obydwu stron zawsze udawało nam się stworzyć zadowalające wszystkich rozwiązanie. Przy współpracy
z firmą fundamnetalny było podejście przy użyciu sprintów. Pozwalało nam to na stopniowe przedstawianie postępów firmie, oraz każdorazowe
otrzymywanie porad i sugestii od strony klienta. Dzięki temu nie doszło do rezejścia się naszych wizji projektu. 

\section{Praca z technologiami}

\section{Wnioski}
\chapter{Baza danych}

\section{Schemat bazy danych}

\begin{figure}[h!] 
\centering
\includegraphics[width=0.99\textwidth]{resources/local/baza-danych-schemat.png}
\caption{Schemat bazy danych} 
\end{figure}

Legenda kolorów:
\begin{itemize}
    \item Żółte - encje globalne współdzielone pomiędzy scenariuszami
    \item Zielone - encje realizujące relacje wiele do wielu
    \item Białe - główne encje, które są wyłączne dla każdego scenariusza
    \item Czerwone - encje pomocnicze 
    \item Niebieski - encje definiujące uprawnienia
\end{itemize}

\section{Opis encji}

\subsection{Encje scenariuszowe}

\begin{figure}[H]
    \centering
    \begin{minipage}{0.8\textwidth} 
        \begin{framed}
            \noindent\textbf{\large Nazwa techniczna:} \texttt{qds\_scenario} \\
            \textbf{\large Nazwa robocza:} Scenariusz \\
            \textbf{\large Opis:} Reprezentuje scenariusz, centralny element systemu organizujący
            wydarzenia, wątki i obiekty w określonych ramach czasowych.
        \end{framed}
    \end{minipage}
\end{figure}


\begin{table}[H]
    \centering
    \renewcommand{\arraystretch}{1.6}
    \begin{tabular}{|>{\bfseries}l|p{0.7\textwidth}|}
        \hline
        \rowcolor[HTML]{EFEFEF} \textbf{Atrybut} & \textbf{Opis} \\
        \hline
        \texttt{id} & Unikalny identyfikator scenariusza. \\
        \hline
        \texttt{title} & Nazwa scenariusza. \\
        \hline
        \texttt{description} & Opis scenariusza mający przybliżyć najważniejsze informacje na jego temat. \\
        \hline
        \texttt{context} & Kontekst w jakim ma miejsce scenariusz. \\
        \hline
        \texttt{purpose} & Cel przewodni scenariusza. \\
        \hline
        \texttt{event\_duration} & Domyślny czas trwania zdarzenia, wyrażony za pomocą wewnętrznej dla scenariusz jednostki czasu. \\
        \hline
        \texttt{start\_date} & Dokładny moment w czasie rozpoczęcia scenariusza, uwzględniający strefę czasową. \\
        \hline
        \texttt{end\_date} & Dokładny moment w czasie zakończenia scenariusza, uwzględniający strefę czasową. \\
        \hline
        \texttt{creation\_date} & Dokładny moment w czasie utworzenia scenariusza, uwzględniający strefę czasową. \\
        \hline
        \texttt{last\_modification\_date} & Dokładny moment w czasie ostatniej aktualizacji scenariusza, uwzględniający strefę czasową. \\
        \hline
    \end{tabular}
    \caption{Atrybuty encji \texttt{scenariusza}}
\end{table}

\begin{figure}[H]
    \centering
    \begin{minipage}{0.8\textwidth} 
        \begin{framed}
            \noindent\textbf{\large Nazwa techniczna:} \texttt{qds\_scenario\_phase} \\
            \textbf{\large Nazwa robocza:} Faza scenariusza \\
            \textbf{\large Opis:} Reprezentuje logiczną fazę czasową w scenariuszu.
            Fazy pozwalają na podział scenariusza na mniejsze, znaczące okresy, ułatwiając organizację i wizualizację wydarzeń.
        \end{framed}
    \end{minipage}
\end{figure}

\begin{table}[H]
    \centering
    \renewcommand{\arraystretch}{1.6}
    \begin{tabular}{|>{\bfseries}l|p{0.8\textwidth}|}
        \hline
        \rowcolor[HTML]{EFEFEF} \textbf{Atrybut} & \textbf{Opis} \\
        \hline
        \texttt{id} & Unikalny identyfikator fazy scenariusza. \\
        \hline
        \texttt{title} & Nazwa fazy scenariusza. \\
        \hline
        \texttt{description} & Opis fazy scenariusza mający przybliżyć najważniejsze informacje na jej temat. \\
        \hline
        \texttt{color} & Kolor zapisany heksadecymalnie, który zostanie użyty podczas wizualizacji fazy. \\
        \hline
        \texttt{start\_date} & czas rozpoczęcia fazy scenariusza, wyrażony za pomocą wewnętrznej dla scenariusz jednostki czasu. \\
        \hline
        \texttt{end\_date} & czas zakończenia fazy scenariusza, wyrażony za pomocą wewnętrznej dla scenariusz jednostki czasu. \\
        \hline
        \texttt{scenario\_id} & klucz obcy scenariusza, do którego należy faza scenariusza. \\
        \hline
    \end{tabular}
    \caption{Atrybuty encji \texttt{fazy scenariusza}}
\end{table}

Dodatkowe informacje
\begin{itemize}
    \item Fazy nie mogą na siebie nachodzić czasowo
\end{itemize}

\subsection{Encje konfiguracyjne}

\begin{figure}[H]
    \centering
    \begin{minipage}{0.8\textwidth} 
        \begin{framed}
            \noindent\textbf{\large Nazwa techniczna:} \texttt{qds\_user} \\
            \textbf{\large Nazwa robocza:} Użytkownik \\
            \textbf{\large Opis:} Reprezentuje użytkownika systemu. Przechowuje podstawowe informacje o użytkowniku 
            oraz jego rolę systemową.
        \end{framed}
    \end{minipage}
\end{figure}

\begin{table}[H]
    \centering
    \renewcommand{\arraystretch}{1.6}
    \begin{tabular}{|>{\bfseries}l|p{0.8\textwidth}|}
        \hline
        \rowcolor[HTML]{EFEFEF} \textbf{Atrybut} & \textbf{Opis} \\
        \hline
        \texttt{id} & Unikalny identyfikator użytkownika. \\
        \hline
        \texttt{email} & Adres e-mail użytkownika. \\
        \hline
        \texttt{first\_name} & Imię użytkownika \\
        \hline
        \texttt{last\_name} & Nazwisko użytkownika \\
        \hline
        \texttt{role} & Rola użytkownika:
        \begin{itemize}
            \item \texttt{USER} - Zwykły użytkownik.
            \item \texttt{ADMIN} - Administrator systemu.
        \end{itemize} \\
        \hline
    \end{tabular}
    \caption{Atrybuty encji \texttt{użytkownika}}
\end{table}

\begin{figure}[H]
    \centering
    \begin{minipage}{0.8\textwidth} 
        \begin{framed}
            \noindent\textbf{\large Nazwa techniczna:} \texttt{qds\_permission} \\
            \textbf{\large Nazwa robocza:} Permisja \\
            \textbf{\large Opis:} Określa poziomy uprawnień dostępu do scenariusza.
        \end{framed}
    \end{minipage}
\end{figure}

\begin{table}[H]
    \centering
    \renewcommand{\arraystretch}{1.6}
    \begin{tabular}{|>{\bfseries}l|p{0.8\textwidth}|}
        \hline
        \rowcolor[HTML]{EFEFEF} \textbf{Atrybut} & \textbf{Opis} \\
        \hline
        \texttt{id} & Unikalny identyfikator permisji. \\
        \hline
        \texttt{role} & Typ uprawnienia:
        \begin{itemize}
            \item \texttt{AUTHOR} - pełne uprawnienia, włącznie z zarządzaniem dostępem.
            \item \texttt{EDIT} - możliwość modyfikacji scenariusza
            \item \texttt{VIEW} - podgląd scenariusza
        \end{itemize} \\
        \hline
        \texttt{scenario\_id} & klucz obcy scenariusza, do którego odnosi się permisja \\
        \hline
        \texttt{user\_id} & klucz obcy użytkownika, do którego odnosi się permisja \\
        \hline
    \end{tabular}
    \caption{Atrybuty encji \texttt{permisji}}
\end{table}

\begin{figure}[H]
    \centering
    \begin{minipage}{0.8\textwidth} 
        \begin{framed}
            \noindent\textbf{\large Nazwa techniczna:} \texttt{qds\_configuration} \\
            \textbf{\large Nazwa robocza:} Konfiguracja \\
            \textbf{\large Opis:} Przechowuje globalne ustawienia konfiguracyjne systemu. Zawiera domyślne wartości 
            i parametry wpływające na działanie całego systemu.
        \end{framed}
    \end{minipage}
\end{figure}

\begin{table}[H]
    \centering
    \renewcommand{\arraystretch}{1.6}
    \begin{tabular}{|>{\bfseries}l|p{0.8\textwidth}|}
        \hline
        \rowcolor[HTML]{EFEFEF} \textbf{Atrybut} & \textbf{Opis} \\
        \hline
        \texttt{id} & Unikalny identyfikator konfiguracji. \\
        \hline
        \texttt{name} & Nazwa ustawień, unikalna dla scenariusza oraz użytkownika \\
        \hline
        \texttt{conf} & Dane konfiguracji przechowywane w formacie JSON. \\
        \hline
        \texttt{scenario\_id} & klucz obcy scenariusza, do którego odnosi się konfiguracja \\
        \hline
        \texttt{user\_id} & klucz obcy użytkownika, do którego odnosi się konfiguracja \\
        \hline
    \end{tabular}
    \caption{Atrybuty encji \texttt{konfiguracji}}
\end{table}

\subsection{Encje Typów}

\begin{figure}[H]
    \centering
    \begin{minipage}{0.8\textwidth} 
        \begin{framed}
            \noindent\textbf{\large Nazwa techniczna:} \texttt{qds\_object\_type} \\
            \textbf{\large Nazwa robocza:} Typ obiektu \\
            \textbf{\large Opis:} Definiuje typ obiektu w systemie, określając jego podstawowe właściwości
            i ograniczenia. Typy mogą tworzyć hierarchię (dziedziczenie) i definiować zasady globalności obiektów.
        \end{framed}
    \end{minipage}
\end{figure}

\begin{table}[H]
    \centering
    \renewcommand{\arraystretch}{1.6}
    \begin{tabular}{|>{\bfseries}l|p{0.8\textwidth}|}
        \hline
        \rowcolor[HTML]{EFEFEF} \textbf{Atrybut} & \textbf{Opis} \\
        \hline
        \texttt{id} & Unikalny identyfikator typu obiektu. \\
        \hline
        \texttt{title} & Unikalny tytuł typu obiektu. \\
        \hline
        \texttt{description} & Opis typu obiektu. \\
        \hline
        \texttt{color} & Kolor zapisany heksadecymalnie, który zostanie użyty podczas wizualizacji obiektu. \\
        \hline
        \texttt{is\_only\_global} & Wartość logiczna stwierdziająca czy obiekt może być tylko w wątku globalnym \\
        \hline
        \texttt{has\_children} & Wartość logiczna stwierdziająca czy obiekt posiada potomne typy \\
        \hline
        \texttt{is\_base\_type} & Wartość logiczna stwierdziająca czy jest typem podstawowym, tworzonym przy tworzeniu scenariusza \\
        \hline
        \texttt{can\_be\_user} & Wartość logiczna stwierdziająca czy obiekt może być przydzielony do użytkownika \\
        \hline
        \texttt{parent\_id} & klucz obcy typu nadrzędnego, za pomocą którego tworzy się hierarchia \\
        \hline
    \end{tabular}
    \caption{Atrybuty encji \texttt{typu obiektu}}
\end{table}

Dodatkowe informacje
\begin{itemize}
    \item Nie można usunąć typu używanego przez obiekty lub szablony
    \item Usunięcie typu jest możliwe tylko gdy nie ma typów potomnych
\end{itemize}

\begin{figure}[H]
    \centering
    \begin{minipage}{0.8\textwidth} 
        \begin{framed}
            \noindent\textbf{\large Nazwa techniczna:} \texttt{qds\_association\_type} \\
            \textbf{\large Nazwa robocza:} Typ asocjacji \\
            \textbf{\large Opis:} Definiuje typ/rodzaj asocjacji możliwy między obiektami w systemie.
            Określa jakie typy obiektów mogą być połączone danym rodzajem asocjacji.
        \end{framed}
    \end{minipage}
\end{figure}

\begin{table}[H]
    \centering
    \renewcommand{\arraystretch}{1.6}
    \begin{tabular}{|>{\bfseries}l|p{0.8\textwidth}|}
        \hline
        \rowcolor[HTML]{EFEFEF} \textbf{Atrybut} & \textbf{Opis} \\
        \hline
        \texttt{id} & Unikalny identyfikator typu asocjacji. \\
        \hline
        \texttt{description} & Opis asocjacji \\
        \hline
        \texttt{firstObjectTypeId} & Identyfikator pierwszego dozwolonego typu obiektu \\
        \hline
        \texttt{secondObjectTypeId} & Identyfikator drugiego dozwolonego typu obiektu \\
        \hline
    \end{tabular}
    \caption{Atrybuty encji \texttt{typu asocjacji}}
\end{table}

Dodatkowe informacje
\begin{itemize}
    \item Nie można usunąć używanego typu asocjacji
    \item Typ obiektu posiada relacje wiele do wielu z encją scenariusza 
\end{itemize}

\subsection{Encje szablonowe}

\begin{figure}[H]
    \centering
    \begin{minipage}{0.8\textwidth} 
        \begin{framed}
            \noindent\textbf{\large Nazwa techniczna:} \texttt{qds\_object\_template} \\
            \textbf{\large Nazwa robocza:} Szablon obiektu \\
            \textbf{\large Opis:} Reprezentuje szablon obiektu określający jego strukturę atrybutów.
            Definiuje jakie atrybuty i jakiego typu musi posiadać obiekt stworzony na podstawie tego szablonu.
        \end{framed}
    \end{minipage}
\end{figure}

\begin{table}[H]
    \centering
    \renewcommand{\arraystretch}{1.6}
    \begin{tabular}{|>{\bfseries}l|p{0.8\textwidth}|}
        \hline
        \rowcolor[HTML]{EFEFEF} \textbf{Atrybut} & \textbf{Opis} \\
        \hline
        \texttt{id} & Unikalny identyfikator szablonu obiektu. \\
        \hline
        \texttt{title} & Unikalny tytuł szablonu. \\
        \hline
        \texttt{description} & Opis szablonu. \\
        \hline
        \texttt{color} & Kolor zapisany heksadecymalnie, który zostanie użyty podczas wizualizacji obiektu. \\
        \hline
        \texttt{object\_type\_id} & klucz obcy typu obiektu, do którego można przypisać szablon \\
        \hline
    \end{tabular}
    \caption{Atrybuty encji \texttt{szablonu obiektu}}
\end{table}

Dodatkowe informacje
\begin{itemize}
    \item Nie można usunąć szablonu używanego przez obiekt
    \item Szablon obiektu posiada relacje wiele do wielu z encją scenariusza 
\end{itemize}

\begin{figure}[H]
    \centering
    \begin{minipage}{0.8\textwidth} 
        \begin{framed}
            \noindent\textbf{\large Nazwa techniczna:} \texttt{qds\_attribute\_template} \\
            \textbf{\large Nazwa robocza:} Szablon atrybutu obiektu \\
            \textbf{\large Opis:} Reprezentuje szablon obiektu określający jego strukturę atrybutów.
            Definiuje jakie atrybuty i jakiego typu musi posiadać obiekt stworzony na podstawie tego szablonu.
        \end{framed}
    \end{minipage}
\end{figure}

\begin{table}[H]
    \centering
    \renewcommand{\arraystretch}{1.6}
    \begin{tabular}{|>{\bfseries}l|p{0.8\textwidth}|}
        \hline
        \rowcolor[HTML]{EFEFEF} \textbf{Atrybut} & \textbf{Opis} \\
        \hline
        \texttt{id} & Unikalny identyfikator szablonu obiektu. \\
        \hline
        \texttt{name} & Unikalna nazwa atrybutu. \\
        \hline
        \texttt{default\_value} & Wartość domyślna atrybutu \\
        \hline
        \texttt{unit} & Jednostka wartości atrybutu \\
        \hline
        \texttt{role} & Typ wartości atrybutu:
        \begin{itemize}
            \item \texttt{INT} - liczba całkowita
            \item \texttt{STRING} - łańcuch znaków
            \item \texttt{DATE} - data lub czas 
            \item \texttt{BOOL} - wartość logiczna
        \end{itemize} \\
        \hline
        \texttt{object\_template\_id} & Klucz obcy szablonu obiektu, do którego przydzielony jest atrybut \\
        \hline
    \end{tabular}
    \caption{Atrybuty encji \texttt{szablonu atrybutu obiektu}}
\end{table}

Dodatkowe informacje
\begin{itemize}
    \item Usuwany kaskadowo wraz z szablonem obiektu
\end{itemize}

\subsection{Encje dotyczące konkretnego obiektu}

\begin{figure}[H]
    \centering
    \begin{minipage}{0.8\textwidth} 
        \begin{framed}
            \noindent\textbf{\large Nazwa techniczna:} \texttt{qds\_object} \\
            \textbf{\large Nazwa robocza:} Obiekt \\
            \textbf{\large Opis:} Reprezentuje obiekt w scenariuszu, który może być modyfikowany przez wydarzenia.
            Obiekt jest globalny jeśli został utworzony w wątku globalnym, w przeciwnym razie jest lokalny 
            i przypisany do konkretnego wątku.
        \end{framed}
    \end{minipage}
\end{figure}

\begin{table}[H]
    \centering
    \renewcommand{\arraystretch}{1.6}
    \begin{tabular}{|>{\bfseries}l|p{0.8\textwidth}|}
        \hline
        \rowcolor[HTML]{EFEFEF} \textbf{Atrybut} & \textbf{Opis} \\
        \hline
        \texttt{id} & Unikalny identyfikator obiektu. \\
        \hline
        \texttt{name} & Unikalna nazwa obiektu. \\
        \hline
        \texttt{object\_type\_id} & Klucz obcy typu obiektu \\
        \hline
        \texttt{template\_id} & Klucz obcy szablonu obiektu \\
        \hline
        \texttt{scenario\_id} & Klucz obcy scenariusza obiektu \\
        \hline
    \end{tabular}
    \caption{Atrybuty encji \texttt{obiektu}}
\end{table}

Dodatkowe informacje
\begin{itemize}
    \item Obiekt można przypisać do użytkownika systemu
\end{itemize}

\begin{figure}[H]
    \centering
    \begin{minipage}{0.8\textwidth} 
        \begin{framed}
            \noindent\textbf{\large Nazwa techniczna:} \texttt{qds\_association} \\
            \textbf{\large Nazwa robocza:} Asocjacja \\
            \textbf{\large Opis:} Reprezentuje asocjację (relację) między dwoma obiektami w systemie.
            Przechowuje informacje o typie relacji i powiązanych obiektach.
        \end{framed}
    \end{minipage}
\end{figure}

\begin{table}[H]
    \centering
    \renewcommand{\arraystretch}{1.6}
    \begin{tabular}{|>{\bfseries}l|p{0.8\textwidth}|}
        \hline
        \rowcolor[HTML]{EFEFEF} \textbf{Atrybut} & \textbf{Opis} \\
        \hline
        \texttt{id} & Unikalny identyfikator asocjacji. \\
        \hline
        \texttt{association\_type\_id} & Klucz obcy typu asocjacji \\
        \hline
        \texttt{object1\_id} & Klucz obcy pierwszego obiektu \\
        \hline
        \texttt{object2\_id} & Klucz obcy drugiego obiektu \\
        \hline
    \end{tabular}
    \caption{Atrybuty encji \texttt{asocjacji}}
\end{table}

Dodatkowe informacje
\begin{itemize}
    \item Kombinacja (typ asocjacji, obiekt1, obiekt2) musi być unikalna
    \item Usuwana automatycznie gdy usuwany jest którykolwiek z obiektów
\end{itemize}

\begin{figure}[H]
    \centering
    \begin{minipage}{0.8\textwidth} 
        \begin{framed}
            \noindent\textbf{\large Nazwa techniczna:} \texttt{qds\_attribute} \\
            \textbf{\large Nazwa robocza:} Atrybut \\
            \textbf{\large Opis:} Reprezentuje atrybut obiektu w systemie wraz z jego wartością początkową
        \end{framed}
    \end{minipage}
\end{figure}

\begin{table}[H]
    \centering
    \renewcommand{\arraystretch}{1.6}
    \begin{tabular}{|>{\bfseries}l|p{0.8\textwidth}|}
        \hline
        \rowcolor[HTML]{EFEFEF} \textbf{Atrybut} & \textbf{Opis} \\
        \hline
        \texttt{id} & Unikalny identyfikator atrybutu. \\
        \hline
        \texttt{initial\_value} & Wartość początkowa atrybutu \\
        \hline
        \texttt{object\_id} & Klucz obcy obiektu, którego dotyczy atrybut \\
        \hline
        \texttt{attribute\_template\_id} & Klucz obcy szablonu atrybutu \\
        \hline
    \end{tabular}
    \caption{Atrybuty encji \texttt{atrybutu obiektu}}
\end{table}

Dodatkowe informacje
\begin{itemize}
    \item Każdy obiekt posiada dokładnie wszystkie atrybuty zdefiniowane przez typ obiektu
    \item Usuwany kaskadowo z obiektem
\end{itemize}

\subsection{Encje wizualizujące upływ czasu}

\begin{figure}[H]
    \centering
    \begin{minipage}{0.8\textwidth} 
        \begin{framed}
            \noindent\textbf{\large Nazwa techniczna:} \texttt{qds\_thread} \\
            \textbf{\large Nazwa robocza:} Wątek \\
            \textbf{\large Opis:} Reprezentuje wątek w scenariuszu - sekwencję wydarzeń w czasie
        \end{framed}
    \end{minipage}
\end{figure}

\begin{table}[H]
    \centering
    \renewcommand{\arraystretch}{1.6}
    \begin{tabular}{|>{\bfseries}l|p{0.8\textwidth}|}
        \hline
        \rowcolor[HTML]{EFEFEF} \textbf{Atrybut} & \textbf{Opis} \\
        \hline
        \texttt{id} & Unikalny identyfikator wątku. \\
        \hline
        \texttt{title} & Tytuł wątku \\
        \hline
        \texttt{description} & Opis wątku \\
        \hline
        \texttt{is\_global} & Wartość logiczna stwierdziająca czy wątek jest globalny, czy lokalny \\
        \hline
        \texttt{scenario\_id} & Klucz obcy scenariusza \\
        \hline
    \end{tabular}
    \caption{Atrybuty encji \texttt{wątku}}
\end{table}

\begin{figure}[H]
    \centering
    \begin{minipage}{0.8\textwidth} 
        \begin{framed}
            \noindent\textbf{\large Nazwa techniczna:} \texttt{qds\_event} \\
            \textbf{\large Nazwa robocza:} Wydarzenie \\
            \textbf{\large Opis:} Reprezentuje wydarzenie w scenariuszu zachodzące w określonym czasie i wątku.
            Wydarzenia są podstawowym mechanizmem wprowadzania zmian w systemie,
            odpowiadają za modyfikacje atrybutów obiektów i ich asocjacji.
        \end{framed}
    \end{minipage}
\end{figure}

\begin{table}[H]
    \centering
    \renewcommand{\arraystretch}{1.6}
    \begin{tabular}{|>{\bfseries}l|p{0.8\textwidth}|}
        \hline
        \rowcolor[HTML]{EFEFEF} \textbf{Atrybut} & \textbf{Opis} \\
        \hline
        \texttt{id} & Unikalny identyfikator wydarzenia. \\
        \hline
        \texttt{title} & Tytuł wydarzenia \\
        \hline
        \texttt{description} & Opis wydarzenia \\
        \hline
        \texttt{event\_time} & czas wydarzenia, wyrażony za pomocą wewnętrznej dla scenariusz jednostki czasu. \\
        \hline
        \texttt{event\_type} & Typ wydarzenia
        \begin{itemize}
            \item \texttt{GLOBAL} - Wydarzenie w wątku globalnym, nadpisuje zmiany z innych wątków
            \item \texttt{NORMAL} - Standardowe wydarzenie w wątku
            \item \texttt{START} - Początek wątku
            \item \texttt{END} - Koniec wątku
            \item \texttt{IDLE} - Wydarzenie puste
            \item \texttt{JOIN\_OUT} - Początek wątku powstałego z połączenia
            \item \texttt{JOIN\_IN} - Koniec wątku wchodzącego do połączenia
            \item \texttt{FORK\_OUT} - Początek wątku powstałego z podziału
            \item \texttt{FORK\_IN} - Wydarzenie w wątku przed podziałem
        \end{itemize} \\
        \hline
        \texttt{thread\_id} & Klucz obcy wątku, w ramach którego ma miejsce wydarzenie \\
        \hline
        \texttt{branching\_id} & Identyfikator powiązanego rozgałęzienia (tylko dla FORK/JOIN) \\
        \hline
    \end{tabular}
    \caption{Atrybuty encji \texttt{wydarzenia}}
\end{table}

\begin{figure}[H]
    \centering
    \begin{minipage}{0.8\textwidth} 
        \begin{framed}
            \noindent\textbf{\large Nazwa techniczna:} \texttt{qds\_attribute\_change} \\
            \textbf{\large Nazwa robocza:} Zmiana atrybutu \\
            \textbf{\large Opis:} Reprezentuje zmianę wartości atrybutu w kontekście wydarzenia.
            Jest częścią historii zmian atrybutów, gdzie każda zmiana jest powiązana
            z konkretnym wydarzeniem (QdsEvent) określającym kiedy nastąpiła.
        \end{framed}
    \end{minipage}
\end{figure}

\begin{table}[H]
    \centering
    \renewcommand{\arraystretch}{1.6}
    \begin{tabular}{|>{\bfseries}l|p{0.8\textwidth}|}
        \hline
        \rowcolor[HTML]{EFEFEF} \textbf{Atrybut} & \textbf{Opis} \\
        \hline
        \texttt{id} & Unikalny identyfikator zmiany atrybutu. \\
        \hline
        \texttt{value} & Nowa wartość atrybutu \\
        \hline
        \texttt{attribute\_id} & Klucz obcy atrybutu obiektu, którego wartość ulega zmianie \\
        \hline
        \texttt{event\_id} & Klucz obcy wydarzenia, w ramach którego odbywa się zmiana atrybutu \\
        \hline
    \end{tabular}
    \caption{Atrybuty encji \texttt{zmiany atrybutu obiektu}}
\end{table}

Dodatkowe informacje
\begin{itemize}
    \item Każda zmiana musi być unikalna dla pary (wydarzenie, atrybut)
    \item Zmiany są usuwane kaskadowo wraz z wydarzeniem lub modyfikowanym atrybutem
\end{itemize}

\begin{figure}[H]
    \centering
    \begin{minipage}{0.8\textwidth} 
        \begin{framed}
            \noindent\textbf{\large Nazwa techniczna:} \texttt{qds\_association\_change} \\
            \textbf{\large Nazwa robocza:} Zmiana asocjacji \\
            \textbf{\large Opis:} Reprezentuje zmianę stanu asocjacji (dodanie/usunięcie) w określonym momencie czasu.
            Jest częścią historii zmian asocjacji, gdzie każda zmiana jest powiązana z konkretnym
            wydarzeniem określającym kiedy i w jakim kontekście nastąpiła.
        \end{framed}
    \end{minipage}
\end{figure}

\begin{table}[H]
    \centering
    \renewcommand{\arraystretch}{1.6}
    \begin{tabular}{|>{\bfseries}l|p{0.8\textwidth}|}
        \hline
        \rowcolor[HTML]{EFEFEF} \textbf{Atrybut} & \textbf{Opis} \\
        \hline
        \texttt{id} & Unikalny identyfikator zmiany asocjacji. \\
        \hline
        \texttt{association\_operation} & Rodzaj operacji
        \begin{itemize}
            \item \texttt{INSERT} - Dodanie nowej asocjacji między obiektami
            \item \texttt{DELETE} - Usunięcie istniejącej asocjacji między obiektami
        \end{itemize} \\
        \hline
        \texttt{association\_id} & Klucz obcy atrybutu obiektu, którego wartość ulega zmianie \\
        \hline
        \texttt{event\_id} & Klucz obcy wydarzenia, w ramach którego odbywa się zmiana atrybutu \\
        \hline
    \end{tabular}
    \caption{Atrybuty encji \texttt{zmiany asocjacji obiektu}}
\end{table}

Dodatkowe informacje
\begin{itemize}
    \item Każda zmiana musi być unikalna dla pary (wydarzenie, asocjacja)
    \item Zmiany są usuwane kaskadowo wraz z wydarzeniem lub modyfikowaną asocjacją
\end{itemize}

\section{Indeksy}

Indeks to struktura danych, pozwalająca na przyśpieszenie operacji odczytu w bazie danych.
Poprzez wygenerowaną strukturę danych, operacja szukania odpowiednich rekordów posiada o wiele mniejszą złożoność.
W naszym przypadku używamy indeksów HASH, ponieważ najlepiej sprawdzają się w operacjach równościowych.
Klucze są unikalne, co znacznie zmniejsza prawdopodobieństwo kolizji. Dodatkowo, kolejność
nie jest istotna w kolumnach identyfikujących.

\begin{itemize}
    \item \textbf{Tabela:} \texttt{qds\_scenario\_phase}, \textbf{Kolumna:} \texttt{scenario\_id}  
    Indeks optymalizujący wyszukiwanie faz scenariusza po identyfikatorze scenariusza.

    \item \textbf{Tabela:} \texttt{qds\_association\_type}, \textbf{Kolumna:} \texttt{first\_object\_type\_id}  
    Indeks przyspieszający wyszukiwanie asocjacji po pierwszym typie obiektu.

    \item \textbf{Tabela:} \texttt{qds\_association\_change}, \textbf{Kolumna:} \texttt{association\_id}  
    Indeks optymalizujący wyszukiwanie zmian danej asocjacji.

    \item \textbf{Tabela:} \texttt{qds\_attribute\_change}, \textbf{Kolumna:} \texttt{event\_id}  
    Indeks przyspieszający wyszukiwanie zmian atrybutów związanych z określonym wydarzeniem.

    \item \textbf{Tabela:} \texttt{qds\_object}, \textbf{Kolumna:} \texttt{scenario\_id}  
    Indeks zapewniający unikalność obiektów w scenariuszu.

    \item \textbf{Tabela:} \texttt{qds\_thread}, \textbf{Kolumna:} \texttt{scenario\_id}, \textbf{Warunek:} \texttt{is\_global = TRUE}  
    Unikalny indeks zapewniający, że w danym scenariuszu może istnieć tylko jeden wątek globalny.

    \item \textbf{Tabela:} \texttt{qds\_thread}, \textbf{Kolumna:} \texttt{scenario\_id}  
    Indeks optymalizujący wyszukiwanie wątków scenariusza.

    \item \textbf{Tabela:} \texttt{qds\_branching}, \textbf{Kolumna:} \texttt{scenario\_id}  
    Indeks przyspieszający wyszukiwanie rozgałęzień danego scenariuszu.

    \item \textbf{Tabela:} \texttt{qds\_event}, \textbf{Kolumna:} \texttt{thread\_id}  
    Indeks optymalizujący wyszukiwanie wydarzeń danego wątku.
\end{itemize}




















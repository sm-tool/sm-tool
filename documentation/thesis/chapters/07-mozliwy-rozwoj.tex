\chapter{Możliwy przyszły rozwój aplikacji}

W tym rozdziale wskazano potencjalne kierunki rozwoju aplikacji. Wymienione funkcjonalności wpłynełyby pozytywnie za równo na
możliwości użytkownika, jakość tworzonych scenariuszy oraz łatwość współpracy.

\section{Perspektywy}

Wizualizacja danych z różnych punktów widzenia:
\begin{itemize}
    \item Koordynator - wizualizacja całościowej realizacji scenariusza na poziomie faz i wątków. Dobrym sposobem wizualizacji tej
    perspektywy jest wykres Gantta.
    \item Aktor - wizualizacja wątków i obiektów istotnych dla konkretnego aktora.
    \item Obserwator w konkretnej lokalizacji - wizualizacja wątków i obiektów istotnych dla całego scenariusza oraz konkretnej fizycznej lokalizacji.
\end{itemize}

Ułatwiłoby to użytkownikom systemu zrozumienie ich roli, oraz ograniczyło zakres scenariusza do informacji im potrzebnych.

\section{Zdarzenia alternatywne}

W naszej implementacji zdarzenie 
może wpłynąć na asocjacje oraz atrybuty obiektów tylko w jeden wybrany sposób. Zdarzenia alternatywne polegałyby na 
zapoczątkowaniu kilku nowych alternatywnych ścieżek fabuły. 
\noindent Jako przykład można wskazać dwie sytuacje:
\begin{itemize}
    \item Uczniowie ewakuowali się na dach.
    \item Uczniowie ewakuowali się do piwnicy.
\end{itemize}
\noindent Po takim zdarzeniu tworzyłyby się nowe alternatywne wątki. Różni się to od aktualnej implementacji wątków, 
ponieważ obecnie umożliwiają one ukazanie równoległych, a nie alternatywnych, zdarzeń.

\section{Moduł ewaluacji ćwiczeń}

Moduł ewaluacji umożliwiłby realizację kolejnego etapu ćwiczeń, czyli analizę i ocenę odbytych ćwiczeń. Opierałby się na 
wykorzystaniu perspektywy obserwatora, który mógłby prowadzić ocenę na poziomie wykonania poszczególnych akcji. 
Wynikiem tych działań byłby całościowy raport oceniający skuteczność przeprowadzonych ćwiczeń,
wskazujący miejsca do poprawy oraz wyciągnięte wnioski.


\section{Zarządzanie wielodostępem}

Przydatny byłby mechanizm proponowania zmian dla właściciela scenariusza. Użytkownik z wyłącznym prawem do odczytu mógłby proponować 
zmiany właścicielowi scenariusza. Dodatkowo, dobrze współgrałby z tym mechanizm perspektywy. Podzielono by scenariusz na części, 
za które odpowiedzialni by byli różni użytkownicy. Prawa odczytu i zmiany dotyczyłyby części scenariusza, a nie tak jak aktualnie 
całości.

\section{Eksport scenariusza}

Dużym usprawnieniem byłaby możliwość eksportu scenariusza do szeroko używanego formatu pliku jak np. XML.
Ułatwiłoby to wymianę informacji pomiędzy organizacjami oraz systemami. Takie rozwiązanie wspiera automatyzację procesów,
oraz zgodność z panującymi standardami.
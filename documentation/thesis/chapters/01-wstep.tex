
\chapter{Wstęp}

\section{Motywacja}
Celem projektu jest opracowanie systemu umożliwiającego tworzenie scenariuszy ćwiczeń sytuacji kryzysowych zgodnie z Trial Guidance Methodology (TGM).

\section{Cel i zakres pracy}
Organizacje zajmujące się ćwiczeniami oraz reagowaniem na kryzysowe sytuacje systematycznie przeprowadzają symulacje oraz testy procedur postępowania, tak aby cyklicznie ulepszać systemy bezpieczeństwa oraz trenować radzenie sobie w sytuacjach niekontrolowanych. Takie działania wymagają aktywnej postawy wielu osób m.in.: służb mundurowych, personelu medycznego, pracowników infrastruktury transportowej oraz użytkowników testowanej aplikacji.
Biorąc pod uwagę złożoność, koszty oraz zasoby takich działań, kluczowe jest ich dokładne ich rozplanowanie z uwzględnieniem obligatoryjnych praktyk takich jak TGM (Trial Guidance Methodology) czy CWA 18009:2023.
Brakuje jednak elastycznego narzędzia wspierającego proces planowania takich eksperymentów, zwłaszcza w
aspekcie projektowania scenariuszy sytuacji.

W związku z tym, potrzebne jest kompleksowe narzędzie umożliwiające tworzenie takich scenariuszy, które będą zawierały infromacje na temat sekwencji działań, poszczególnych akcji, wątków, obiektów oraz atrybutów. W rezultacie rozwiązanie będzie wspierać każdego uczestnika procesu od twórcy scenariusza, po poszczególnych aktorów, którzy odgrywają role w scenariuszu. Wymogiem jest stworzenie wizualizacji tego rozwiązania, która będzie czytelna i będzie pozwalać na współpracę wielu twórców modelu w trakcie jego projektowania i umożliwiać spojrzenie
na scenariusz z różnych punktów widzenia np. organizatora i koordynatora ćwiczeń, aktorów
odgrywających zaplanowane role, obserwatorów i ekspertów oceniających przebieg ćwiczenia.
Narzędzie powinno być zgodne z Trial Guidance Methodology (TGM), opisaną w książce \emph{Trial Guidance Methodology Handbook} \cite{tgmhandbook}.

W celu zaplanowania scenariusza powinniśmy uzyskać dwie informacje. Pierwszą z nich jest cel, który chcemy osiągnąć, natomiast drugą są okoliczności, które są odpowiedzialne za wyznaczenie granicy działania.
W planowaniu scenariusza sytuacji kryzysowych możemy wyróżnić pewien cel, polegający na identyfikacji luk w systemie zarządzania kryzysowego. Możemy je wyznaczyć przy pomocy osób biorących udział w tego typu działaniach, na przykład przeprowadzić z nimi wywiad.
Jednak trzeba mieć na uwadze, że każdy może mieć doczynienia z innym typem luk. Na przykład osoby dowodzące akcją strażaków będa zwracały uwagę na luki w zarządzaniu zdarzenaimi. Natomiast policjanci drogówki skupią swoją uwagę na wskazaniu problemów w patrolowaniu ulic.
Tak więc każda z luk jest zależna od kilku aspektów takich jak: roli oraz obowiązków osoby, a także otoczenia w jakim ta osoba pracuje. Połączenie tej charakterystyki można nazwać kontekstem scenariusza.

Osoba planująca scenariusz musi w jakiś sposób zidentyfikować wyżej opisane luki w systemie. W tym celu zaleca się skorzystanie z pewnych sposobów, które to umożliwiają. Są to między innymi:
\begin{itemize}
\item Badanie źródeł wtórnych,
\item Przeprowadzenie wywiadów grupowych,
\item Podejście, które jest połączeniem badania źródeł wtórnych oraz wywiadów grupowych,
\item Warsztaty.
\end{itemize}

Po zidentyfikowaniu luk możemy przejść do określenia kontekstu scenariusza. Pomocny w tym może być szablon znajdujący się w Trial Guidance Tool. Dzięki niemu możemy wyznaczć najważniejsze aspekty, na podstawie konkretnej sytuacji, w której dana luka występuje. Sytuacja i scenariusz różnią się między innymi tym, że ten drugi jest pewnym momentem w czasie, w którym znajduje się luka. Dlatego właśnie wskazane jest przeanalizowanie warunków, osób, otoczenia w tym punkcie czasu
i na podstawie wniosków opracować to co jest wymagane w danym scenariuszu i to co jest opcjonalne. Następnym krokiem jest przygotowanie stanu początkowego, który jest mapą aktualnych procesów. W celu jego przygotowania można skorzystać z BPMN, czyli Business Process Modeling and Notation.

\subsection{Metodologia}
System powinien umożliwiać tworzenie i zarządzanie m.in. scenariuszami, obiektami, wątkami oraz akcjami. Na etapie planowania oraz analizy
zdecydowaliśmy się na skorzystanie z diagramów ERD, schematów bazy danych i diagramów kontekstu. Umożliwiło to nam dokładną analizę problemu oraz dokładne rozplanowanie pracy.
System zaprojektowaliśmy tak, żeby używał wzorca Repository Pattern, który zapewnia elastyczność, logiczność, łatwe testowanie, ale także decoupling. W celu wizualizacji architektury wykonaliśmy
m.in. diagramy klas oraz diagramy przepływu danych. Podczas planowania bazy danych uwzględniliśmy zasady dobrej jej orgazniacji. W celu testowania aplikacji napisaliśmy testy jednostkowe, integracyjne oraz funkcjonalne.
Podczas rezalizacji projektu pracowaliśmy w metodyce Scrum. Każdy sprint trwał 2 tygodnie oraz kończył się podsumowaniem sprintu oraz przedstawieniem postępów klientowi. Pracą zarządzaliśmy używając Github Issues, w którym rozpisywaliśmy zadania oraz podczas planowania spritów przydzielaliśmy je do poszczególnych osób.
Dokumentacja techniczna projektu została stworzona przy użyciu Swaggera.

\section{Struktura pracy}
Praca składa się z następujących sekcji:
\begin{description}
    \item[Wstęp] -- wprowadzenie, podstawowe założenia, podział pracy.
    \item[Wymagania] -- wymagania biznesowe, przypadki użycia, wymagania pozafunkcjonalne.
    \item[Architektura] -- decyzje architektoniczne, wykorzystane technologie, markitektura 4+1.
    \item[Implementacja] -- szczegółowy opis implementacji rest api, backendu, frontendu oraz testów.
    \item[Zapewnienie jakości] - Definition of Done, Scrum, Czysty kod, Code review, testy.
    \item[Doświadczenia] - doświadczenia developerów podczas pracy z technologiami, w zespole oraz kontakcie z klientem.
    \item[Możliwy przyszły rozwój aplikacji] - możliwe rozwiązania, które można wdrożyć w przyszłości.
    \item[Zakończenie] - podsumowanie pracy.
\end{description}

\section{Podział pracy}
Projekt został zrealizowany w ramach współpracy w programie Software Development Studio Politechniki Poznańskiej. Dzięki temu mieliśmy okazję pracować ze studentami
specjalizacji Software Engineering. Byli oni naszymi project managerami, a w dodatku odpowiadali za organizację sprintów i pracy. Naszym klientem była firma ITTI, z którą
wspólnie omówiliśmy szczegóły i wymagania projektu podczas pierwszego spotaknia w kwietniu 2024 roku. Następnie na zakończenie każdego sprintu odbywało się spotkanie w formie zdalnej
z przedstawicielami firmy, na których pokazywaliśmy postępy prac, konsultowaliśmy poprawki oraz aktualizowaliśmy wymagania klienta. Główną formą komunikacji z przedstawicielami firmy
była poczta elektroniczna.

W ramach pracy nad projektem \textbf{Kacper Kuras} odpowiadał za projektowanie, modelowanie bazy danych i jej konfigurację. Uwzględniało
to tworzenie skomplikowanych zapytań w warstwie repository oraz plików Entity, które definiowały strukturę bazy danych w Javie. Tworzył rówież część logiki aplikacji, która zapewniała wydajną oraz spójną pracę bazy danych i serwera.

Natomiast \textbf{Bartosz Adamczewski} skupił swoją uwagę na tworzeniu testów, które obejmowały testy jednostkowe oraz integracyjne w celu zapewnienia jakości produktu.
Do jego zadań należała również implementacja websocketów, które umożliwiły dwukierunkowe sesje komunikacyjne między klientem a serwerem. Bartosz odpowiadał za opracowanie części endpointów.

\textbf{Kacper Woźniak} odpowiadał za tworzenie i konfigurację endpointów oraz tworzenie logiki biznesowej aplikacji. Wspierał również pisanie testów, w celu szybszej weryfikacji popraności poszczególnych modułów.
Do obowiązaków Kacpra należała również implementacja i konfiguracja mechanizmu CORS.

\textbf{Jakub Wieczorek} był odpowiedzialny za projektowanie oraz implementację warsty wizualnej aplikacji. Swoją uwagę skupił między innymi na konfiguracji komunikacji z serwerem, wydajność aplikacji przeglądarkowej oraz
był odpowiedzialny za stylizację i zarządzanie elementami takimi jak wizualizacja scenariusza na osi czasu przy użyciu skomplikowanych bibliotek frontendowych.

W ramach pracy wspólnej jako zespół zaprojektowaliśmy i omawialiśmy na spotkaniach główne zalożenia pracy oraz wymagania klienta. Całość pracy realizowaliśmy zgodnie z podejściem zwinnym (Agile), dzięki któremu mieliśmy możliwość iteracyjnego wdrażania
kolejnych etapów implementacji oraz systematyczne analizowanie wymagań i ich ewentalnych poprawek. Wstęp i wymagania zostały opracowane przy pomocy materiałów przygotowanych przez inż. Adama Kopca oraz inż. Julię Mularczyk z Software Development Studio.

Procentowy podział pracy przedstawia się następująco:

\begin{table}[h!]
    \centering
    \begin{tabular}{|l|l|l|}
        \hline
        \textbf{Imię i nazwisko} & \textbf{Zakres prac} & \textbf{Udział procentowy}\\
        \hline
        Bartosz Adamczewski & Backend, Testy & 25\%\\
        Kacper Kuras & Baza danych, Backend & 25\%\\
        Jakub Wieczorek & Frontend, DevOps & 25\%\\
        Kacper Woźniak & Backend & 25\%\\
        \hline
    \end{tabular}
    \caption{Zakres prac oraz udział procentowy członków zespołu w projekcie.}
    \label{tab:udzial_zespolu}
\end{table}
